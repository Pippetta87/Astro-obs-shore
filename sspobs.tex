\documentclass[main.tex]{subfiles}
%\usepackage{functions}
\begin{document}
%%%%%%%%%%%%%%%%%%%%%%%%%%%%%%%%%%%%
\pgfkeys{/pgf/fpu=true}
%%%%%%%%%%%%%%%%%%%%%%%%%%%%%%%%%%%% Define Constant for pgf
%\pgfmathparse{} %
%\edef\Gconst{\pgfmathresult}

\chapter{Simulazione FOV o popolazione stellare}

\section{Stellar pop modelling}

Distribuzione di massa con gaussiana $\mu, \sigma$: genero popolazione di N stelle con distribuzione di massa determinata considerando il numero di stelle aspettato nell'intervallo di massa con valore determinato casualmente con distro poissoniana.

\section{Octave functions}

\begin{itemize}
\item distribuzione massa stelle (e luminosit\'a)

\lstset{language={matlab},breakatwhitespace=true, breaklines=true}
\begin{lstlisting}
    function [mdcv popstar LF]=IMFd(Nstar)
#IMF=(m/minmass)^(-2)
#(m,i): somma massa*numero stelle=RV=frazione massa iniziale che ha formato stelle/et\'a cluster
#linspace(mmin,x) vettore lunghezza definita da dm: mtot/mmin??
#->f(m(i))*Nstar
ndm=round(sqrt(Nstar))+1;
#md=linspace(0,1,ndm);#mass interval discretization. Linear
#md=logspace(1,log(1/(2*ndm)),ndm);#mass interval discretization. Log
#mdcv=[0 logspace(-1,1,ndm)(1:end-1)]+diff([0 logspace(-1,1,ndm)/2]);#valori rappresentativi segmento
mdcv=linspace(0,1,ndm)(1:end-1)+diff(linspace(0,1,ndm)/2);#valori rappresentativi segmento
#sum(10.^(linspace(0,1,1000))-logspace(0,1,1000))=0
#mrp=(md+1/(2*Nstar)).^-2;#relative probability star with given mass
#deltam=diff([0 logspace(-1,1,ndm)]);
deltam=diff(linspace(0,1,ndm))
mrp=(mdcv).^-2.*deltam;#relative probability star with given mass
prob=mrp/sum(mrp);
starexp=prob*Nstar;
popstar=poissrnd(starexp);
LF=linspace(0,1,ndm);#provvisorio
#0.23(m)^2.3 M<0.43Msun
#m^4: 0.43<m<2
#1.4m^3.5: 2<m<20
#32000m: m>55msun
endfunction
\end{lstlisting}

\item Posizione stelle FOV
    
    
\end{itemize}

\section{Schema dell'esperienza}

%decodearray={0.2 0.5},


\edef\frequno{20e6}    %RF freq uno
\edef\freqdue{16.9e6}    %RF freq due
\edef\Iuno{200e-3}    %corrente minimo corrente di picco uno
\edef\Idue{170e-3}    %corrente minimo corrente di picco due
 ($\nu_{ris}/I\propto\pgfmathparse{\frequno/\Iuno}\pgfmathprintnumber{\pgfmathresult}$, $\nu_{ris}/I\propto\pgfmathparse{\freqdue/\Idue}\pgfmathprintnumber{\pgfmathresult}$).


\pgfkeys{/pgf/fpu=false}

\end{document}
